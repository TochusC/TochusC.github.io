%%%%%%%%%%%%%%%%%%%%%%%%%%%%%%%%%%%%%%%%%%%%%%%%%%%
{\Large\textbf{许祖耀}}\quad \quad \quad \quad \contactInfo{15753183270(微信)}{205329624@qq.com}{https://github.com/TochusC}
%%%%%%%%%%%%%%%%%%%%%%%%%%%%%%%%%%%%%%%%%%%%%%%%%%%
%%%%%%%%%%%%%%%%%%%%%%%%%%%%%%%%%%%%%%%%%%%%%%%%%%%
\logosection{\faGraduationCap}{教育经历}

\datedline{\textbf{中国石油大学(华东)\quad 211 \quad 双一流 }\quad 计算机科学与技术 \quad 本科}{\dateRange{2021.09}{至今}}

\begin{itemize}
  \item \textbf{学习情况:} 前五学期排名:16.36\%(18/110),平均学分绩:87.83
  \item  \textbf{相关课程:}程序设计C/C++(100),程序设计Java(100),移动互联网实践(98),程序设计实习(97),数据结构与算法(95),数据库课程设计(95),机器学习(93),编译原理(92),计算机图形学(91),大学英语(4学期均90+)
  \item  \textbf{个人荣誉:}综合优秀奖学金*2,校优秀学生*2,校优秀团员,中共党员
   \item  \textbf{英语成绩:}CET-4: \textbf{558}, CET-6: \textbf{559}
\end{itemize}
%%%%%%%%%%%%%%%%%%%%%%%%%%%%%%%%%%%%%%%%%%%%%%%%%%%

%%%%%%%%%%%%%%%%%%%%%%%%%%%%%%%%%%%%%%%%%%%%%%%%%%%
\logosection{\faTrophy}{竞赛情况} 
\datedline{“中国软件杯”大学生程序设计竞赛(A04龙源风电功率预测系统开发)\textbf{国家三等奖} \quad 团队负责人}{2023.07}
\datedline{全国大学生数学建模竞赛(B题海域测深)\quad 省一等奖 \quad 建模兼代码}{2023.10}
\datedline{美国大学生数学建模竞赛(B题搜索潜水艇) \quad Honorable Mention \quad 建模兼代码}{2024.02}
\datedline{中国高校计算机大赛-网络技术挑战赛(SDN监控平台) \quad 省二等奖}{2023.06}
%%%%%%%%%%%%%%%%%%%%%%%%%%%%%%%%%%%%%%%%%%%%%%%%%%%

%%%%%%%%%%%%%%%%%%%%%%%%%%%%%%%%%%%%%%%%%%%%%%%%%%%
\logosection{\faWrench}{项目经历}

\datedline{\textbf{云龙风电-风电实时预测系统}}{\dateRange{2023.03}{2023.06}}
\datedline{\biInfo{时序预测神经网络、可视化实时预测}{竞赛项目}}{https://github.com/TochusC/windpower-forecast-system}
使用TCN+LSTM+MLP多层复合神经网络预测风机发电功率,利用TCN+LSTM处理历史数据,MLP综合历史数据向量和天气预报来预测风机发电功率,准确率排东部赛区\textbf{第三名}并获得国家三等奖。项目使用前后端分离方式,搭建了实时预测网站系统,神经网络使用PaddlePaddle深度学习框架搭建,\textbf{项目由自己独立完成}。

\vspace{8pt}
\datedline{\textbf{通慧智教-AI赋能的智能教学辅助平台}}{\dateRange{2024.03}{2024.04}}
\datedline{\biInfo{自然语言交互、Unity VR开发}{竞赛项目}}{https://github.com/TochusC/ai-assistant-teaching-website}
通过Unity引擎和虚拟人技术实现AI虚拟助理,使用Prompt工程扩展语言模型的应用场景,将用户的自然语言输入,转换为网站、虚拟人的操纵指令;让用户可以通过语音或文字方式操控网页、进行路由跳转、后端数据调取总结,完成相应业务逻辑,同时提供了通过SteamVR套件开发的UnityVR桌面程序,\textbf{项目由自己独立完成}。

\vspace{8pt}
\datedline{\textbf{Pybicc类C语言编译器}}{\dateRange{2024.03}{2024.04}}
\datedline{\biInfo{图形化界面的C编译器与汇编语言解释器}}{https://github.com/TochusC/pybicc}
通过Python实现的类C语言编译器,能够将C语言代码编译为Intel 80x86汇编指令, 编译器通过词法分析、语法分析、语义分析生成汇编代码;汇编语言解释器按字节模拟数据存取,并解释执行汇编指令实时得出运算结果;同时提供图形化界面,支持显示词法分析、语法分析中间结果, \textbf{项目由自己独立完成}。

\vspace{8pt}
\datedline{\textbf{Ource操作系统}}{\dateRange{2022.10}{2022.10}}
\datedline{\biInfo{32位Intel i486操作系统}}{https://github.com/TochusC/ource}
参考自川合秀实编著,周自恒翻译的《30天自制操作系统》中的Haribote系统编写,由C与汇编语言实现,通过多级优先队列调度算法进行任务调度、采用时间片轮转方式分配处理器资源,使用分段存储管理进行内存管理、利用FAT12文件系统进行文件管理。 \textbf{项目由自己独立完成}。


\vspace{8pt}
\datedline{\textbf{基于AlexNet的医疗眼疾识别}}{\dateRange{2023.11}{2024.01}}
\datedline{\biInfo{医疗影像识别、深度学习神经网络}{}}{暂未上传}
使用PaddlePaddle深度学习框架复现AlexNet,并应用在百度大脑和中山大学中山眼科中心提供的眼疾识别数据集iChallenge-PM上,模型识别准确度达到93\%,模型的搭建、训练、测试\textbf{均由自己独立完成}。

\vspace{8pt}
\datedline{\textbf{AtomSounds氛围乐编辑器}}{\dateRange{2023.11}{2024.01}}
\datedline{\biInfo{安卓应用开发、客户端-服务器通信架构}{}}{暂未上传}
使用Android Studio开发的安卓应用程序,程序支持用户通过拖拽原子声音元素自定制环境氛围乐。程序采用CS的通信架构:使用自己的笔记本进行内网穿透作为服务器,通过HTTP协议与移动安卓客户端通信,实现数据的存储、共享、更新,\textbf{项目由自己独立完成}。



%%%%%%%%%%%%%%%%%%%%%%%%%%%%%%%%%%%%%%%%%%%%%%%%%%%


%%%%%%%%%%%%%%%%%%%%%%%%%%%%%%%%%%%%%%%%%%%%%%%%%%%
\logosection{\faCogs}{专业技能}
\begin{itemize}[parsep=0.5ex]
  \item 熟练使用Git、Docker等开发工具,项目能力及开发能力较强;系统学习了GAMES101(图形学)和深度学习。
  \item 熟练掌握Unity(VR)开发,Blender三维建模,Android Studio安卓开发,PyQT桌面程序开发、OpenGL等技术,
  \item 部分其他项目已上传至GitHub仓库,同时相关介绍视频也已上传至BiliBili:https://space.bilibili.com/10478211
\end{itemize}
%%%%%%%%%%%%%%%%%%%%%%%%%%%%%%%%%%%%%%%%%%%%%%%%%%%


%%%%%%%%%%%%%%%%%%%%%%%%%%%%%%%%%%%%%%%%%%%%%%%%%%%
\logosection{\faInfo}{其他}
\begin{itemize}[parsep=0.5ex]
  \item 首先感谢您拨冗阅读我的简历。\\

  \item 在本科期间我并没有很明确的目标性...我出于纯粹的兴趣爱好和一部分的就业目的学习了Unity开发、VR开发,三维建模、图形学等技术,并参加了若干个限时游戏开发竞赛(GameJam),独立制作了多个Unity小游戏。\\

  \item 在学生工作上,我参与了学校双创部门,协助了多项创新创业赛事的举办,帮助布置赛场、管理人员、制定规则等,但更多的还有一些端茶倒水的琐碎小事。
  \vspace{8pt}\\
 在社团活动中,我积极参加了机器人社团,并学习掌握了Arduino等单片机开发技能,在其中我使用Wifiduino独立搭建拼装了一个可用手机操控的智能车,并代表社团进行了外出展览。
  \vspace{8pt}\\
  在体育锻炼方面,我加入了学校的定向越野队,随队进行每日训练,并参加定向越野比赛取得名次;此外还有很多活动和经历,它们可能并没有很大的意义和作用,但对于每一件事情我都尽力做到了最好。\\

  \item 我认为自己有着比较强的内驱力,自学能力也很好;相比于身边的其他就业导向的同学,我的计算机基础知识更为坚实牢固,对于每一门课程,不管教学好坏,我都尽量去学习掌握了相关知识与技能;同时我也认为自己拥有较强的设计和表达能力,喜欢通过编程进行自由创造的感觉。\\
  
  \item 最后请允许我再次感谢您阅读至此,我理解自己与那些985高校、竞赛金牌的优秀同学们相比仍有着很大差距,但“知不足而后进,量山远而力行”,倘若我幸能与你们一起从事,我将有着全力以赴,与其他同学一起齐头并进的\textbf{信心与决心}。
\end{itemize}
%%%%%%%%%%%%%%%%%%%%%%%%%%%%%%%%%%%%%%%%%%%%%%%%%%%